\documentclass[margin,line]{res}
% -------------------------------------------------------------------------
\usepackage[usenames,dvipsnames]{xcolor}
\usepackage[unicode=true,colorlinks=true,linkcolor=blue]{hyperref}
\hypersetup{urlcolor=BlueViolet} % Does not apply color to href's
\hypersetup{colorlinks,urlcolor=BlueViolet} % href's are correct, but navigation

\oddsidemargin -.5in
\evensidemargin -.5in
\textwidth=6.0in
\itemsep=0in
\parsep=0in

\newenvironment{list1}{
    \begin{list}{\ding{113}}{%
        \setlength{\itemsep}{0in}
        \setlength{\parsep}{0in} \setlength{\parskip}{0in}
        \setlength{\topsep}{0in} \setlength{\partopsep}{0in}
        \setlength{\leftmargin}{0.17in}}}{
    \end{list}}
\newenvironment{list2}{
  \begin{list}{$\bullet$}{%
      \setlength{\itemsep}{0in}
      \setlength{\parsep}{0in} \setlength{\parskip}{0in}
      \setlength{\topsep}{0in} \setlength{\partopsep}{0in}
      \setlength{\leftmargin}{0.2in}}}{\end{list}}

% \pagestyle{plain} 
\pagestyle{empty}  % No page #

% -------------------------------------------------------------------------
\begin{document}
\newcommand{\link}[1]{\texttt{#1}}
\providecommand{\tightlist}{%
    \setlength{\itemsep}{0pt}\setlength{\parskip}{0pt}}


% ---------------------------------------------------------------------------
\name{Erik J. Peterson, PhD\vspace*{.1in}}

\begin{resume}
\section{\sc }
\vspace{.05in}

% Info
\begin{tabular}{@{}p{2in}p{4in}}
{\it E-mail:}  erik.exists@gmail.com   & {\it Webpage:} \href{http://robotpuggle.com}{http://robotpuggle.com} \\
\end{tabular}

% \vspace{.1cm}
\section{\sc About me}
I'm a scientist with machine learning expertise. I've worked in both industry and academia. I have experience studying curiosity, play, and open-endedness in reinforcement learning. I am presently focused on designing new systems for automated causal reasoning in complex systems.

% I've studied coordination in biophysical and artificial models.

% I study learning and coordination in distributed systems, as well as exploration in individuals. I am a theorist who blends neuroscience, computer science, and biology.

% I'm broadly interested in curiosity and causality for use in artificial intelligence, and as mathematical ideas. I'm also very interested in unconventional kinds of computation in biology.

% \vspace{-.3cm}
% I am interested in leadership roles -- in industry or academia.

% ---------------------------------------------------------------------------
% \vspace{-.05cm}
\section{\sc Recent Experience}
\vspace{-.2cm}
{\bf Pastuer Labs} - New York, NY\\
{\em Senior Research Scientist} \hfill {\bf 2022 - Present}\\
I am a technical lead building systems for automated causal reasoning on complex, multi-part, problems in physical science. 

\vspace{-.2cm}
{\bf Carnegie Mellon University} - Pittsburgh, PA \\
{\em Research Fellow (Research Scientist)} \hfill {\bf 2018 - 2022}\\
I developed a mathematical accounts of play and curiosity for use in reinforcement learning (\href{https://github.com/CoAxLab/infomercial}{Github}) and multi-agent systems (\href{https://github.com/parenthetical-e/parkid}{Github}). I also established new fundamental limits for astrocyte computation. 

\vspace{-.2cm}
{\bf Kernel, LLC} - Los Angeles, CA\\
{\em Research Scientist} \hfill {\bf 2017 - 2018}\\
I was the technical lead building a system for complex spatio-temporal field shaping in deep brain stimulation. This project blended biophysical modeling with deep neural networks and led to 400,000 fold speed up -- a key requirement for \emph{real-time} use.

\vspace{-.2cm}
{\bf U.C. San Diego} - San Diego, CA\\
{\em Postdoctoral Fellow} \hfill {\bf 2014 - 2017}\\
I conducted theoretical research on the coding properties of neural oscillations. I also co-lead development of a python tool to analyze electrophysiological data which has found widespread use in the neuroscience community.

% ---------------------------------------------------------------------------
\vspace{-.2cm}
\section{\sc Education}
{\bf Colorado State University}, Fort Collins, CO\\
%{\em Department of Statistics}
\vspace*{-.15in}
\begin{list1}
    \tightlist
    \item[] Ph.D, Psychology \hfill {\bf 2012}
\end{list1}

\vspace*{-.15in}
{\bf California Polytechnic State University}, San Luis Obispo, CA\\
%{\em Department of Mathematics and Statistics}
\vspace*{-.15in}
\begin{list1}
    \tightlist
    \item[] B.S., Chemistry and Biochemistry; Minor, Philosophy \hfill {\bf May 2004}
\end{list1}


% ---------------------------------------------------------------------------
\vspace{-.1cm}
\section{\sc Programming}
\vspace{-.1cm}
{\bf Python}\\
\vspace*{-.15in}
\begin{list1}
    \tightlist
    \item[] Core ML - Linear methods to deep neural nets - \{\emph{pytorch}, \emph{ray}, \emph{sklearn}\} \hfill {\bf Expert}
\end{list1}

\vspace{-.5cm}
{\bf R} \\
\vspace*{-.2in}
\begin{list1}
    \tightlist
    \item[] Core DS - Visualization, analysis, and statistical testing - \{\emph{tidyverse}\} \hfill {\bf Expert}
\end{list1}



% ---------------------------------------------------------------------------
\vspace{-.1cm}
\section{\sc Projects}
\vspace{-.1cm}
{\bf The Exploration Book} (\href{https://github.com/parenthetical-e/explorations-book}{Github}) \\
Authoring a book on exploration in biology, ranging from random search, to reinforcement learning, to curiosity, imagination, and reasoning. I developed a python package (\href{https://github.com/parenthetical-e/explorationlib}{Github}) to make it easy to explore exploration. 

% ---------------------------------------------------------------------------
\vspace{-.2cm}
\section{\sc Press \& Public Talks}
Brain's `Background Noise' May Hold Clues to Persistent Mysteries, \emph{Quanta Magazine}, 2021. \\
Build Your Own Brainwaves, \emph{Nerd Nite}, Los Angeles, Feb 2018. \\
Conflicted Data Science, \emph{Open San Diego}, San Diego, Feb, 2016. \\
In Theory You're Paying Attention, \emph{Ignite}, San Diego, Nov 2016. \\
    
% ---------------------------------------------------------------------------
\vspace{-.2cm} 
\section{\sc Select publications}
Donoghue T*, Haller M*, \textbf{Peterson EJ}*, et al, Parameterizing Neural Power Spectra into Periodic and Aperiodic Components, \emph{Nature Neuroscience} 23 1655-1665 (2020). [*]: Co-first. 
\\ 
\vspace{-.35cm} 
\\
\textbf{Peterson EJ} \& Verstynen T, Curiosity eliminates the exploration-exploitation dilemma, \emph{bioRxiv} 671362v8 (2020). 

% ---------------------------------------------------------------------------
\end{resume}
\end{document}
