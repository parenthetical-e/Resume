\documentclass[margin,line]{res}
% -------------------------------------------------------------------------
\usepackage[usenames,dvipsnames]{xcolor}
\usepackage[unicode=true,colorlinks=true,linkcolor=blue]{hyperref}
\hypersetup{urlcolor=BlueViolet} % Does not apply color to href's
\hypersetup{colorlinks,urlcolor=BlueViolet} % href's are correct, but navigation

\oddsidemargin -.5in
\evensidemargin -.5in
\textwidth=6.0in
\itemsep=0in
\parsep=0in

\newenvironment{list1}{
    \begin{list}{\ding{113}}{%
        \setlength{\itemsep}{0in}
        \setlength{\parsep}{0in} \setlength{\parskip}{0in}
        \setlength{\topsep}{0in} \setlength{\partopsep}{0in}
        \setlength{\leftmargin}{0.17in}}}{
    \end{list}}
\newenvironment{list2}{
  \begin{list}{$\bullet$}{%
      \setlength{\itemsep}{0in}
      \setlength{\parsep}{0in} \setlength{\parskip}{0in}
      \setlength{\topsep}{0in} \setlength{\partopsep}{0in}
      \setlength{\leftmargin}{0.2in}}}{\end{list}}

% \pagestyle{plain} 
\pagestyle{empty}  % No page #

% -------------------------------------------------------------------------
\begin{document}
\newcommand{\link}[1]{\texttt{#1}}
\providecommand{\tightlist}{%
    \setlength{\itemsep}{0pt}\setlength{\parskip}{0pt}}


% ---------------------------------------------------------------------------
\name{Erik J. Peterson, PhD\vspace*{.1in}}

\begin{resume}
\section{\sc }
\vspace{.05in}

% Info
\begin{tabular}{@{}p{2in}p{4in}}
{E-mail:}  erik.exists@gmail.com   & {Webpage:} \href{http://robotpuggle.com}{http://robotpuggle.com} \\
\end{tabular}

\vspace{-.2cm}
\section{\sc In summary}
Excellent scientist. Thoughtful engineer.
% I'm a scientist and engineer with expertise in artificial intelligence, reinforcement learning, neuroscience, and natural computation. In industry and academia I have designed and led high-risk, high-reward research at the intersection of biology, engineering, and computing. 

%I'm a scientist with machine learning expertise. I've worked in both industry and academia. I have experience studying curiosity, play, and open-endedness in reinforcement learning. I am presently focused on designing new systems for automated causal reasoning in complex systems.

% I've studied coordination in biophysical and artificial models.

% I study learning and coordination in distributed systems, as well as exploration in individuals. I am a theorist who blends neuroscience, computer science, and biology.

% I'm broadly interested in curiosity and causality for use in artificial intelligence, and as mathematical ideas. I'm also very interested in unconventional kinds of computation in biology.

% \vspace{-.3cm}
% I am interested in leadership roles -- in industry or academia.

% ---------------------------------------------------------------------------
% \vspace{-.05cm}
\section{\sc Experience}
\vspace{-.1cm}
{\bf Pastuer Labs} - New York, NY\\
{\em Staff Scientist -- Advanced Projects Lead} \hfill {\bf 2023 - Present}\\
Project lead overseeing internal and external development of all data-driven multi-physics models for advanced applications in Industrial Cyber-Physical Systems (ICPS). Project lead causal discovery methods for ICPS applications (production and research).

\vspace{-.2cm}
{\em Senior Research Scientist} \hfill {\bf 2022 - 2023}\\
Co-lead scientific machine learning research (Neural Operators, Graph Neural Networks, etc). Lead developer scientific machine learning models (production and research). Lead developer causal analysis methods for physical systems. Established “simulation intelligence” methods that exploit physics to do distributed analog computations.

\vspace{-.1cm}
{\bf Carnegie Mellon University} - Pittsburgh, PA \\
{\em Research fellow (Scientist)} \hfill {\bf 2018 - 2022}\\
Developed mathematical accounts of play and curiosity for use in deep reinforcement learning (\href{https://github.com/CoAxLab/infomercial}{Github}) and multi-agent systems (\href{https://github.com/parenthetical-e/parkid}{Github}). This work answered a fundamental question in decision science (the explore-exploit dilemma). Established new theoretical upper limit for biological computation.

\vspace{-.1cm}
{\bf Kernel, LLC} - Los Angeles, CA\\
{\em Senior Research Scientist} \hfill {\bf 2017 - 2018}\\
Developed a system for complex spatio-temporal field shaping in deep brain stimulation. This project blended \textit{biophysical models} with artificial neural networks that led to a 400,000 fold speed-up -- a key requirement for real-time (production) use.

\vspace{-.1cm}
{\bf U.C. San Diego} - San Diego, CA\\
{\em Postdoctoral Fellow} \hfill {\bf 2014 - 2017}\\
Theoretical and computational research on the coding properties of neural oscillations. Co-lead development of a python tool to analyze electrophysiological data which has found \emph{widespread} use in the neuroscience community (\href{https://github.com/fooof-tools/fooof}{SpecParam}).

% ---------------------------------------------------------------------------
\vspace{-.2cm}
\section{\sc Education}
{\bf Colorado State University} (Fort Collins, CO) --  Ph.D, Psychology; Masters, Psychology.\\
%{\em Department of Statistics}
\vspace*{-.15in}
% \begin{list1}
%     \tightlist
%     \item[] Ph.D, Psychology \hfill% {\bf 2012}
% \end{list1}

\vspace*{-.15in}
{\bf California Polytechnic State University} (San Luis Obispo, CA) -- B.S., Chemistry; B.S. Biochemistry; Minor, Philosophy.\\
%{\em Department of Mathematics and Statistics}
% \vspace*{-.15in}
% \begin{list1}
%     \tightlist
%     \item[]  \hfill% {\bf May 2004}
% \end{list1}


% ---------------------------------------------------------------------------
\vspace{-.6cm}
\section{\sc Programming} Developed production-ready machine learning models in modern frameworks (jax, torch). Expert scientific programmer (python). Fluent in standard tools (git, docker, etc).

% \vspace{-.1cm}
% {\bf Python}\\
% \vspace*{-.15in}
% \begin{list1}
%     \tightlist
%     \item[] Core ML - Linear models to deep neural networks - \{\emph{pytorch}, \emph{jax}, \emph{sklearn}\} \hfill {\bf Expert}
% \end{list1}

% \vspace{-.5cm}
% {\bf R} \\
% \vspace*{-.2in}
% \begin{list1}
%     \tightlist
%     \item[] Core DS - Visualization, analysis, and statistical testing - \{\emph{tidyverse}\} \hfill {\bf Expert}
% \end{list1}


% ---------------------------------------------------------------------------
% \vspace{-.1cm}
% \section{\sc Projects}
% \vspace{-.1cm}
% {\bf The Exploration Book} (\href{https://github.com/parenthetical-e/explorations-book}{Github}) \\
% Authoring a book on exploration in biology, ranging from random search, to reinforcement learning, to curiosity, imagination, and reasoning. I developed a python package (\href{https://github.com/parenthetical-e/explorationlib}{Github}) to make it easy to explore exploration. 

% ---------------------------------------------------------------------------
\vspace{-.2cm}
\section{\sc Press \& Public Talks}
Brain's `Background Noise' May Hold Clues to Persistent Mysteries, \emph{Quanta Magazine}, 2021. \\
Build Your Own Brainwaves, \emph{Nerd Nite}, Los Angeles, Feb 2018. \\
% Conflicted Data Science, \emph{Open San Diego}, San Diego, Feb, 2016. \\
In Theory You're Paying Attention, \emph{Ignite}, San Diego, Nov 2016. \\
    
% ---------------------------------------------------------------------------
\vspace{-.5cm} 
\section{\sc Select publications}
\textbf{Peterson EJ} \& Lavin A, Physical computing for materials acceleration platforms, Matter 5, 3586-3596 (2022).
\\ 
\vspace{-.35cm} 
\\
Donoghue T*, Haller M*, \textbf{Peterson EJ}*, et al, Parameterizing Neural Power Spectra into Periodic and Aperiodic Components, \emph{Nature Neuroscience} 23 1655-1665 (2020). [*]: Co-first. 
\\ 
\vspace{-.35cm} 
\\
\textbf{Peterson EJ} \& Verstynen T, Curiosity eliminates the exploration-exploitation dilemma, \emph{bioRxiv} 671362v8 (2020). 
\\ 
\vspace{-.35cm} 
\\
\textbf{Peterson EJ}, What can astrocytes compute?, \emph{bioRxiv} 465192 (2021).

% ---------------------------------------------------------------------------
\end{resume}
\end{document}
