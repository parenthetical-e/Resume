\documentclass[margin,line]{res}
% -------------------------------------------------------------------------
\usepackage[usenames,dvipsnames]{xcolor}
\usepackage[unicode=true,colorlinks=true,linkcolor=blue]{hyperref}
\hypersetup{urlcolor=BlueViolet} % Does not apply color to href's
\hypersetup{colorlinks,urlcolor=BlueViolet} % href's are correct, but navigation

\oddsidemargin -.5in
\evensidemargin -.5in
\textwidth=6.0in
\itemsep=0in
\parsep=0in

\newenvironment{list1}{
    \begin{list}{\ding{113}}{%
        \setlength{\itemsep}{0in}
        \setlength{\parsep}{0in} \setlength{\parskip}{0in}
        \setlength{\topsep}{0in} \setlength{\partopsep}{0in}
        \setlength{\leftmargin}{0.17in}}}{
    \end{list}}
\newenvironment{list2}{
  \begin{list}{$\bullet$}{%
      \setlength{\itemsep}{0in}
      \setlength{\parsep}{0in} \setlength{\parskip}{0in}
      \setlength{\topsep}{0in} \setlength{\partopsep}{0in}
      \setlength{\leftmargin}{0.2in}}}{\end{list}}

% \pagestyle{plain} 
\pagestyle{empty}  % No page #

% -------------------------------------------------------------------------
\begin{document}
\newcommand{\link}[1]{\texttt{#1}}
\providecommand{\tightlist}{%
    \setlength{\itemsep}{0pt}\setlength{\parskip}{0pt}}


% ---------------------------------------------------------------------------
\name{Erik J. Peterson, PhD\vspace*{.1in}}

\begin{resume}
\section{\sc }
\vspace{.05in}

% Info
\begin{tabular}{@{}p{2in}p{4in}}
{E-mail:}  erik.exists@gmail.com   & {Webpage:} \href{http://robotpuggle.com}{http://robotpuggle.com} \\
\end{tabular}

\vspace{-.2cm}
\section{\sc In summary}
Excellent scientist. Thoughtful engineer. 
% I'm a scientist and engineer with expertise in artificial intelligence, reinforcement learning, neuroscience, and natural computation. In industry and academia I have designed and led high-risk, high-reward research at the intersection of biology, engineering, and computing. 

%I'm a scientist with machine learning expertise. I've worked in both industry and academia. I have experience studying curiosity, play, and open-endedness in reinforcement learning. I am presently focused on designing new systems for automated causal reasoning in complex systems.

% I've studied coordination in biophysical and artificial models.

% I study learning and coordination in distributed systems, as well as exploration in individuals. I am a theorist who blends neuroscience, computer science, and biology.

% I'm broadly interested in curiosity and causality for use in artificial intelligence, and as mathematical ideas. I'm also very interested in unconventional kinds of computation in biology.

% \vspace{-.3cm}
% I am interested in leadership roles -- in industry or academia.

% ---------------------------------------------------------------------------
% \vspace{-.05cm}
\section{\sc Experience}
\vspace{-.1cm}
{\bf Pasteur Labs} - New York, NY\\
{\em Advanced Projects Lead} \hfill {\bf 2023 - Present}\\
Lead overseeing internal and external development of scientific machine learning for all advanced applications in physical and industrial systems (production and research). Project lead causal discovery methods for industrial applications (production and research). 

\vspace{-.2cm}
{\em Senior Scientist} \hfill {\bf 2022 - 2023}\\
Co-lead scientific machine learning research program focused on neural operators, graph neural networks, etc. Lead developer scientific machine learning (research and production). Lead developer causal analysis methods for physical systems (research). Established “simulation intelligence” methods that use physics to do distributed analog computations (research). 

\vspace{-.1cm}
{\bf Carnegie Mellon University} - Pittsburgh, PA \\
{\em Research Fellow (Scientist)} \hfill {\bf 2018 - 2022}\\
Developed mathematical accounts of curiosity in reinforcement learning (\href{https://github.com/CoAxLab/infomercial}{Github}) and multi-agent systems (\href{https://github.com/parenthetical-e/parkid}{Github}) which reframed and answered a key open question in decision science. Established a new theoretical upper limit for biological computation (research). 

\vspace{-.1cm}
{\bf Kernel, LLC} - Los Angeles, CA\\
{\em Senior Scientist} \hfill {\bf 2017 - 2018}\\
Developed a model for complex spatio-temporal electrical field shaping for use in brain computer interfaces and deep brain stimulation. This project blended biophysical models with deep neural networks and led to a 400,000 fold speed-up -- a key requirement for real-time (production) use.

\vspace{-.1cm}
{\bf U.C. San Diego} - San Diego, CA\\
{\em Postdoctoral Fellow} \hfill {\bf 2014 - 2017}\\
Conducted theoretical and computational research on the optimal coding properties of neural oscillations (research). Co-lead development of a python tool to analyze electrophysiological data which has found widespread use in the neuroscience community (\href{https://github.com/fooof-tools/fooof}{SpecParam}).

% -- OPTIONAL
% \vspace{-.1cm}
% {\bf Colorado State University} - Fort Collins, CO\\
% {\em Graduate Research Assistant} \hfill {\bf 2006 - 2012}\\
% Reinforcement learning, category learning, and fMRI data analysis (research). 

% \vspace{-.1cm}
% {\bf Biosearch Technologies} - Novato, CA\\
% {\em Research Assistant II} \hfill {\bf 2004 - 2006}\\
% Optimization of high-throughput chemistry (production); reporter gene development (research).

% ---------------------------------------------------------------------------
\vspace{-.2cm}
\section{\sc Education}
{\bf Colorado State University} (Fort Collins, CO) --  Ph.D, Psychology; Masters, Psychology.\\
\vspace*{-.15in}

\vspace*{-.15in}
{\bf California Polytechnic State University} (San Luis Obispo, CA) -- B.S., Chemistry; B.S. Biochemistry; Minor, Philosophy.\\

% ---------------------------------------------------------------------------
\vspace{-.5cm}
\section{\sc Programming} Developed production-ready machine learning models in modern frameworks (jax, torch). Expert scientific programmer (python). Fluent in standard tools (git, docker, etc).

% ---------------------------------------------------------------------------
\vspace{-.2cm}
\section{\sc Press \& Public Talks}
Brain's `Background Noise' May Hold Clues to Persistent Mysteries, \emph{Quanta Magazine}, 2021. \\
Build Your Own Brainwaves, \emph{Nerd Nite}, Los Angeles, Feb 2018. \\
% Conflicted Data Science, \emph{Open San Diego}, San Diego, Feb, 2016. \\
In Theory You're Paying Attention, \emph{Ignite}, San Diego, Nov 2016. \\
    
% ---------------------------------------------------------------------------
\vspace{-.5cm} 
\section{\sc Select publications}

\textbf{Peterson EJ} \& Lavin A, Physical Computing for Materials Acceleration Platforms, Matter 5, 3586-3596 (2022).
\\ 
\vspace{-.35cm} 
\\
% Lavin A, Zenil H, Paige B, Krakauez D, Gottschlich J, Mattson T, Anandkumar A, Choudry S, Rocki K, Baydin A.G, Prunkl C, Paige B, Isayev O, \textbf{Peterson EJ}, McMahon PL, Macke J, Cranmer K, Zhang J, Wainwright H, Hanuka A, Veloso M, Assefa S, Zheng S, Pfeffer A, Simulation Intelligence: Towards a New Generation of Scientific Methods, \emph{arXiv} 2112.03235 (2021).
% \\ 
% \vspace{-.35cm} 
% \\
\textbf{Peterson EJ}, What Can Astrocytes Compute?, \emph{bioRxiv} 465192 (2021).
\\ 
\vspace{-.35cm} 
\\
Donoghue T*, Haller M*, \textbf{Peterson EJ}*, et al, Parameterizing Neural Power Spectra into Periodic and Aperiodic Components, \emph{Nature Neuroscience} 23 1655-1665 (2020). [*]: Co-first. 
\\ 
\vspace{-.35cm} 
\\
\textbf{Peterson EJ} \& Verstynen T, Curiosity Eliminates the Exploration-Exploitation Dilemma, \emph{bioRxiv} 671362v8 (2020). 
% \\ 
% \vspace{-.35cm} 
% \\
% \textbf{Peterson EJ} \& Voytek B, Homeostatic Mechanisms May Shape the Type and Duration of Oscillatory Modulation, \emph{J Neurophys} 124[1] (2020).
% \\ 
% \vspace{-.35cm} 
% \\
% Gao RD, \textbf{Peterson EJ}, Voytek B, Inferring Synaptic Excitation/Inhibition Balance from Field Potentials, \emph{Neuroimage} 158 (2017).


% ---------------------------------------------------------------------------
\end{resume}
\end{document}