\documentclass[margin,line]{res}
% -------------------------------------------------------------------------
\usepackage[usenames,dvipsnames]{xcolor}
\usepackage[unicode=true,colorlinks=true,linkcolor=blue]{hyperref}
\hypersetup{urlcolor=BlueViolet} % Does not apply color to href's
\hypersetup{colorlinks,urlcolor=BlueViolet} % href's are correct, but navigation

\oddsidemargin -.5in
\evensidemargin -.5in
\textwidth=6.0in
\itemsep=0in
\parsep=0in

\newenvironment{list1}{
    \begin{list}{\ding{113}}{%
        \setlength{\itemsep}{0in}
        \setlength{\parsep}{0in} \setlength{\parskip}{0in}
        \setlength{\topsep}{0in} \setlength{\partopsep}{0in}
        \setlength{\leftmargin}{0.17in}}}{
    \end{list}}
\newenvironment{list2}{
  \begin{list}{$\bullet$}{%
      \setlength{\itemsep}{0in}
      \setlength{\parsep}{0in} \setlength{\parskip}{0in}
      \setlength{\topsep}{0in} \setlength{\partopsep}{0in}
      \setlength{\leftmargin}{0.2in}}}{\end{list}}

% \pagestyle{plain} 
\pagestyle{empty}  % No page #

% -------------------------------------------------------------------------
\begin{document}
\newcommand{\link}[1]{\texttt{#1}}
\providecommand{\tightlist}{%
    \setlength{\itemsep}{0pt}\setlength{\parskip}{0pt}}


% ---------------------------------------------------------------------------
\name{Erik J. Peterson, PhD\vspace*{.1in}}

\begin{resume}
\section{\sc }
\vspace{.05in}

% Info
\begin{tabular}{@{}p{2in}p{4in}}
{E-mail:}  erik.exists@gmail.com   & {Webpage:} \href{http://robotpuggle.com}{http://robotpuggle.com} \\
\end{tabular}

\vspace{-.2cm}
\section{\sc About me}
Excellent scientist. Thoughtful software engineer.
% I'm a scientist and engineer with expertise in artificial intelligence, reinforcement learning, neuroscience, and natural computation. In industry and academia I have designed and led high-risk, high-reward research at the intersection of biology, engineering, and computing. 

%I'm a scientist with machine learning expertise. I've worked in both industry and academia. I have experience studying curiosity, play, and open-endedness in reinforcement learning. I am presently focused on designing new systems for automated causal reasoning in complex systems.

% I've studied coordination in biophysical and artificial models.

% I study learning and coordination in distributed systems, as well as exploration in individuals. I am a theorist who blends neuroscience, computer science, and biology.

% I'm broadly interested in curiosity and causality for use in artificial intelligence, and as mathematical ideas. I'm also very interested in unconventional kinds of computation in biology.

% \vspace{-.3cm}
% I am interested in leadership roles -- in industry or academia.

% ---------------------------------------------------------------------------
% \vspace{-.05cm}
\section{\sc Recent Experience}
\vspace{-.1cm}
{\bf Pastuer Labs} - New York, NY\\
{\em Staff Scientist -- Advanced Projects Lead} \hfill {\bf 2023 - Present}\\
Technical lead overseeing internal and external development of multi-scale and multi-physics models of complex physical systems. Technical lead developing causal discovery methods for Industrial Cyber-Physical Systems (ICPS; Ongoing). Theoretical research on causation in complex physical systems.

\vspace{-.2cm}
{\em Senior Research Scientist} \hfill {\bf 2022 - 2023}\\
Lead developer scientific machine learning models (production and research). Technical co-lead scientific machine learning research. Technical lead in developing causal analysis methods for ICPS. “Simulation intelligence” methods for physical computation.

\vspace{-.1cm}
{\bf Carnegie Mellon University} - Pittsburgh, PA \\
{\em Research Scientist} \hfill {\bf 2018 - 2022}\\
Developed a mathematical accounts of play and curiosity for use in Deep Reinforcement Learning (\href{https://github.com/CoAxLab/infomercial}{Github}) and Multi-Agent systems (\href{https://github.com/parenthetical-e/parkid}{Github}). Developed new theoretical limits for astrocyte computation. 

\vspace{-.1cm}
{\bf Kernel, LLC} - Los Angeles, CA\\
{\em Senior Research Scientist} \hfill {\bf 2017 - 2018}\\
Technical lead building a system for complex spatio-temporal field shaping in deep brain stimulation. This project blended biophysical modeling with artificial neural networks and led to 400,000 fold speed-up -- a key requirement for \emph{real-time} use.

\vspace{-.1cm}
{\bf U.C. San Diego} - San Diego, CA\\
{\em Postdoctoral Fellow} \hfill {\bf 2014 - 2017}\\
Theoretical and computational research on the coding properties of neural oscillations. Co-lead development of a python tool (\href{https://github.com/fooof-tools/fooof}{SpecParam}) to analyze electrophysiological data which has found \emph{widespread} use in the neuroscience community.

% ---------------------------------------------------------------------------
\vspace{-.2cm}
\section{\sc Education}
{\bf Colorado State University} (Fort Collins, CO) --  Ph.D, Psychology; Masters, Psychology.\\
%{\em Department of Statistics}
\vspace*{-.15in}
% \begin{list1}
%     \tightlist
%     \item[] Ph.D, Psychology \hfill% {\bf 2012}
% \end{list1}

\vspace*{-.15in}
{\bf California Polytechnic State University} (San Luis Obispo, CA) -- B.S., Chemistry; B.S. Biochemistry; Minor, Philosophy.\\
%{\em Department of Mathematics and Statistics}
% \vspace*{-.15in}
% \begin{list1}
%     \tightlist
%     \item[]  \hfill% {\bf May 2004}
% \end{list1}


% ---------------------------------------------------------------------------
\vspace{-.6cm}
\section{\sc Programming} I am an experienced scientific programmer (python). I have developed production-ready machine learning models in modern frameworks (jax, torch). Fluent in standard tools (git, docker, etc).

% \vspace{-.1cm}
% {\bf Python}\\
% \vspace*{-.15in}
% \begin{list1}
%     \tightlist
%     \item[] Core ML - Linear models to deep neural networks - \{\emph{pytorch}, \emph{jax}, \emph{sklearn}\} \hfill {\bf Expert}
% \end{list1}

% \vspace{-.5cm}
% {\bf R} \\
% \vspace*{-.2in}
% \begin{list1}
%     \tightlist
%     \item[] Core DS - Visualization, analysis, and statistical testing - \{\emph{tidyverse}\} \hfill {\bf Expert}
% \end{list1}


% ---------------------------------------------------------------------------
% \vspace{-.1cm}
% \section{\sc Projects}
% \vspace{-.1cm}
% {\bf The Exploration Book} (\href{https://github.com/parenthetical-e/explorations-book}{Github}) \\
% Authoring a book on exploration in biology, ranging from random search, to reinforcement learning, to curiosity, imagination, and reasoning. I developed a python package (\href{https://github.com/parenthetical-e/explorationlib}{Github}) to make it easy to explore exploration. 

% ---------------------------------------------------------------------------
\vspace{-.2cm}
\section{\sc Press \& Public Talks}
Brain's `Background Noise' May Hold Clues to Persistent Mysteries, \emph{Quanta Magazine}, 2021. \\
Build Your Own Brainwaves, \emph{Nerd Nite}, Los Angeles, Feb 2018. \\
Conflicted Data Science, \emph{Open San Diego}, San Diego, Feb, 2016. \\
In Theory You're Paying Attention, \emph{Ignite}, San Diego, Nov 2016. \\
    
% ---------------------------------------------------------------------------
\vspace{-.5cm} 
\section{\sc Select publications}
\textbf{Peterson EJ} \& Lavin A, Physical computing for materials acceleration platforms, Matter 5, 3586-3596 (2022).
\\ 
\vspace{-.35cm} 
\\
Donoghue T*, Haller M*, \textbf{Peterson EJ}*, et al, Parameterizing Neural Power Spectra into Periodic and Aperiodic Components, \emph{Nature Neuroscience} 23 1655-1665 (2020). [*]: Co-first. 
\\ 
\vspace{-.35cm} 
\\
\textbf{Peterson EJ} \& Verstynen T, Curiosity eliminates the exploration-exploitation dilemma, \emph{bioRxiv} 671362v8 (2020). 
\\ 
\vspace{-.35cm} 
\\
\textbf{Peterson EJ}, What can astrocytes compute?, \emph{bioRxiv} 465192 (2021).

% ---------------------------------------------------------------------------
\end{resume}
\end{document}
