\documentclass[margin,line]{res}
% -------------------------------------------------------------------------
\usepackage[usenames,dvipsnames]{xcolor}
\usepackage[unicode=true,colorlinks=true,linkcolor=blue]{hyperref}
\hypersetup{urlcolor=BlueViolet} % Does not apply color to href's
\hypersetup{colorlinks,urlcolor=BlueViolet} % href's are correct, but navigation

\oddsidemargin -.5in
\evensidemargin -.5in
\textwidth=6.0in
\itemsep=0in
\parsep=0in

\newenvironment{list1}{
    \begin{list}{\ding{113}}{%
        \setlength{\itemsep}{0in}
        \setlength{\parsep}{0in} \setlength{\parskip}{0in}
        \setlength{\topsep}{0in} \setlength{\partopsep}{0in}
        \setlength{\leftmargin}{0.17in}}}{
    \end{list}}
\newenvironment{list2}{
  \begin{list}{$\bullet$}{%
      \setlength{\itemsep}{0in}
      \setlength{\parsep}{0in} \setlength{\parskip}{0in}
      \setlength{\topsep}{0in} \setlength{\partopsep}{0in}
      \setlength{\leftmargin}{0.2in}}}{\end{list}}

% \pagestyle{plain} 
\pagestyle{empty}  % No page #

% -------------------------------------------------------------------------
\begin{document}
\newcommand{\link}[1]{\texttt{#1}}
\providecommand{\tightlist}{%
    \setlength{\itemsep}{0pt}\setlength{\parskip}{0pt}}


% ---------------------------------------------------------------------------
\name{Erik J. Peterson, PhD\vspace*{.1in}}

\begin{resume}
\section{\sc }
\vspace{.05in}

\begin{tabular}{@{}p{2in}p{2in}p{2in}}
{E-mail:}  {\href{mailto:erik.exists@gmail.com}{erik.exists@gmail.com}} & {Website:} \href{http://robotpuggle.com}{robotpuggle.com} & {Github:} \href{https://github.com/parenthetical-e/}{@parenthetical-e} \\
\end{tabular}

\vspace{-.4cm}
\section{\sc In summary}
I have worked and published in scientific machine learning, causal analysis, chemistry,  biochemistry, nanotechnology, surface science, computational neuroscience, reinforcement learning, and biological computation. I have deployed machine learning models to production. And I once spilled \$200k in chemicals on the floor of my lab. \textbf{Excellent scientist, thoughtful engineer, and I learn from my mistakes}.

% Research leader. Excellent scientist. Thoughtful engineer. 
% In industry and academia I have worked and published in scientific machine learning, causal analysis, chemistry, nanotechnology, (computational) neuroscience, biochemistry, reinforcement learning, and biological computation. I am an excellent scientist and a thoughtful software engineer. 

% Excellent scientist. Thoughtful engineer. 

%---------------------------------------------------------------------------
\vspace{-.35cm}
\section{\sc Experience}
{\bf Phinyx} - Providence, RI\\
{\em Principle Scientist} \hfill {\bf 2024 - Current}\\
Head of research, automated programming for scientific computing. Led the team. Wrote production code. 

\vspace{-.25cm}
{\bf Pasteur Labs} - New York, NY\\
{\em Staff Scientist, Advanced Projects Lead} (final position) \hfill {\bf 2022 - 2024}\\
Led projects in causal AI and scientific machine learning. Focus was bridging academic research with industrial demands. Wrote a comprehensive scientific ML library ($>$30 networks). Did new research in  "simulation intelligence" methods for analog computation with physical systems.

% \vspace{-.35cm}
% {\em Senior Scientist} \hfill {\bf 2022 - 2023}\\

\vspace{-.25cm}
{\bf Carnegie Mellon University} - Pittsburgh, PA\\
{\em Research Fellow} \hfill {\bf 2019 - 2022}\\
Did new research on mathematical models of curiosity in reinforcement learning; established a new theoretical upper limit for biological computation.

\vspace{-.25cm}
{\bf Kernel} - Los Angeles, CA\\
{\em Senior Scientist} \hfill {\bf 2017 - 2018}\\
Led team developing model for complex spatio-temporal electrical field shaping, achieving 400,000-fold speed-up for real-time use in brain-computer interfaces.

\vspace{-.25cm}
{\bf U.C. San Diego} - San Diego, CA\\
{\em Postdoctoral Fellow} \hfill {\bf 2014 - 2017}\\
Conducted theoretical and computational research on the optimal coding properties of neural oscillations. Co-developed of a python tool to analyze electrophysiological time-series which has found widespread use in the neuroscience community and been downloaded $>$275,000 times.

\vspace{-.25cm}
{\bf Colorado State University} - Fort Collins, CO\\
{\em Graduate Research Assistant} \hfill {\bf 2006 - 2012}

\vspace{-.25cm}
{\bf Biosearch Technologies} - Novato, CA\\
{\em Research Assistant II} \hfill {\bf 2004 - 2006}\\
Optimized high-throughput chemistry for DNA synthesis; developed reporter genes.

\vspace{-.25cm}
\section{\sc Education}
{\bf Colorado State University} (Fort Collins) - Ph.D, Psychology; Masters, Psychology.\\

\vspace*{-.3in}
{\bf California Polytechnic State University} (San Luis Obispo, CA) -- B.S., Chemistry; B.S., Biochemistry; Minor, Philosophy.\\

% ---------------------------------------------------------------------------
\vspace{-.8cm}
\section{\sc Programming} Developed production-ready machine learning models in modern frameworks (jax, torch). Expert scientific programmer (python). Fluent in standard development tools (git, docker, etc).

% ---------------------------------------------------------------------------
% \vspace{-.35cm}
% \section{\sc Press/Talks}
% Brain's `Background Noise' May Hold Clues to Persistent Mysteries, \emph{Quanta Magazine}, 2021. \\
% Build Your Own Brainwaves, \emph{Nerd Nite}, Los Angeles, Feb 2018. \\
% % Conflicted Data Science, \emph{Open San Diego}, San Diego, Feb, 2016. \\
% In Theory You're Paying Attention, \emph{Ignite}, San Diego, Nov 2016. \\
    
% ---------------------------------------------------------------------------
\vspace{-.4cm} 
\section{\sc Select publications.}
\textsc{Total citations}: $>$2,000. \textsc{h-index}: 14.
\\ 
\vspace{-.4cm} 
\\
\textbf{Peterson EJ} \& Lavin A, Physical Computing for Materials Acceleration Platforms, \textit{Matter} 5, 3586-3596 (2022).
% -- LONG FORMAT -- 
\\ 
\vspace{-.35cm} 
\\
Lavin A, et al, Simulation Intelligence: Towards a New Generation of Scientific Methods, \emph{arXiv} 2112.03235 (2021).
% \\ 
% \vspace{-.35cm} 
% \\
% \textbf{Peterson EJ}, What Can Astrocytes Compute?, \emph{bioRxiv} 465192 (2021).
\\ 
\vspace{-.35cm} 
\\
Donoghue T*, Haller M*, \textbf{Peterson EJ}*, et al, Parameterizing Neural Power Spectra into Periodic and Aperiodic Components, \emph{Nature Neuroscience} 23 1655-1665 (2020). 
% \\ 
% \vspace{-.35cm} 
% \\
% \textbf{Peterson EJ} \& Verstynen T, Curiosity Eliminates the Exploration-Exploitation Dilemma, \emph{bioRxiv} 671362v8 (2020). 
% -- LONG FORMAT -- 
% \\ 
% \vspace{-.35cm} 
% \\
% \textbf{Peterson EJ} \& Voytek B, Homeostatic Mechanisms May Shape the Type and Duration of Oscillatory Modulation, \emph{J Neurophys} 124[1] (2020).
% \\ 
% \vspace{-.35cm} 
% \\
% Gao RD, \textbf{Peterson EJ}, Voytek B, Inferring Synaptic Excitation/Inhibition Balance from Field Potentials, \emph{Neuroimage} 158 (2017).

% ---------------------------------------------------------------------------
\end{resume}
\end{document}