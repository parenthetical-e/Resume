\documentclass[margin,line]{res}
% -------------------------------------------------------------------------
\usepackage[usenames,dvipsnames]{xcolor}
\usepackage[unicode=true,colorlinks=true,linkcolor=blue]{hyperref}
\hypersetup{urlcolor=BlueViolet} % Does not apply color to href's
\hypersetup{colorlinks,urlcolor=BlueViolet} % href's are correct, but navigation

\oddsidemargin -.5in
\evensidemargin -.5in
\textwidth=6.0in
\itemsep=0in
\parsep=0in

\newenvironment{list1}{
    \begin{list}{\ding{113}}{%
        \setlength{\itemsep}{0in}
        \setlength{\parsep}{0in} \setlength{\parskip}{0in}
        \setlength{\topsep}{0in} \setlength{\partopsep}{0in}
        \setlength{\leftmargin}{0.17in}}}{
    \end{list}}
\newenvironment{list2}{
  \begin{list}{$\bullet$}{%
      \setlength{\itemsep}{0in}
      \setlength{\parsep}{0in} \setlength{\parskip}{0in}
      \setlength{\topsep}{0in} \setlength{\partopsep}{0in}
      \setlength{\leftmargin}{0.2in}}}{\end{list}}

% \pagestyle{plain} 
\pagestyle{empty}  % No page #

% -------------------------------------------------------------------------
\begin{document}
\newcommand{\link}[1]{\texttt{#1}}
\providecommand{\tightlist}{%
    \setlength{\itemsep}{0pt}\setlength{\parskip}{0pt}}

% ---------------------------------------------------------------------------
\name{Erik J. Peterson, PhD\vspace*{.1in}}

\begin{resume}
\section{\sc }
\vspace{.05in}

\begin{tabular}{@{}p{2in}p{2in}p{2in}}
{E-mail:}  {\href{mailto:erik.exists@gmail.com}{erik.exists@gmail.com}} & {Website:} \href{http://robotpuggle.com}{robotpuggle.com} & {Github:} \href{https://github.com/parenthetical-e/}{@parenthetical-e} \\
\end{tabular}

\vspace{-.4cm}
\section{\sc In summary}
An interdisciplinary scientist-engineer who has worked and published in scientific machine learning, causal analysis, chemistry, nanotechnology, neuroscience, and reinforcement learning, while building production systems and learning from his many mistakes (like that one time with the \$200k sample).

%---------------------------------------------------------------------------
\vspace{-.25cm}
% \section{\sc Experience}
\section{\sc Experience}
{\bf Atomic Machines} - Berkeley, CA\\
{\em Principal AI Engineer} \hfill {\bf 2024 - Current}\\
Leading development of AI-powered platform for manufacturing advanced devices. Key research areas: using LLMs to (re)formulate problems into formats fit for classical optimization; building automated testing agents for numerical solution validation; automated NLP-driven programming of device-scale simulations.

\vspace{-.25cm}
{\bf Phinyx} - Providence, RI\\
{\em Principal Scientist} \hfill {\bf 2024}\\
Head of research, automated programming for scientific computing. Rewrote core library for NLP-driven program synthesis to make it work with any outside library or language. Did new research on chain-of-thought methods to scale Phinyx's technology to large industry-scale simulations.

\vspace{-.25cm}
{\bf Pasteur Labs} - New York, NY\\
{\em Staff Scientist, Advanced Projects Lead} (final position) \hfill {\bf 2022 - 2024}\\
Led projects in causal AI and scientific machine learning. Focus was bridging academic research with industrial demands. Wrote a comprehensive scientific machine learning library ($>$30 networks). Did new research on using physics to do analog computation.

% \vspace{-.35cm}
% {\em Senior Scientist} \hfill {\bf 2022 - 2023}\\

\vspace{-.25cm}
{\bf Carnegie Mellon University} - Pittsburgh, PA\\
{\em Research Fellow} \hfill {\bf 2019 - 2022}\\
Did new research on mathematical models of curiosity in reinforcement learning; established a new theoretical upper limit for biological computation.

\vspace{-.25cm}
{\bf Kernel} - Los Angeles, CA\\
{\em Senior Scientist} \hfill {\bf 2017 - 2018}\\
Led team developing model for complex spatio-temporal electrical field shaping, achieving 400,000-fold speed-up for real-time use in brain-computer interfaces.

\vspace{-.25cm}
{\bf U.C. San Diego} - San Diego, CA\\
{\em Postdoctoral Fellow} \hfill {\bf 2014 - 2017}\\
Did new research on optimal coding in neural oscillations. Co-developed new software to analyze electrophysiological time-series that is widely used in neuroscience (downloaded $>$275,000 times).

% \vspace{-.25cm}
% {\bf Colorado State University} - Fort Collins, CO\\
% {\em Graduate Research Assistant} \hfill {\bf 2006 - 2012}\\
% Did new research in human reinforcement learning. \emph{Thesis}: Rewards are categories?

% \vspace{-.25cm}
% {\bf Biosearch Technologies} - Novato, CA\\
% {\em Research Assistant II} \hfill {\bf 2004 - 2006}\\
% Optimized high-throughput chemistry for DNA synthesis; developed reporter genes.

\vspace{-.3cm}
\section{\sc Education}
{\bf Colorado State University} (Fort Collins) - Ph.D., Psychology; Master's, Psychology. Thesis: \emph{Rewards are Categories?}\\

\vspace*{-.3in}
{\bf California Polytechnic State University} (San Luis Obispo, CA) -- B.S., Chemistry; B.S., Biochemistry; Minor, Philosophy.\\

% ---------------------------------------------------------------------------
\vspace{-.8cm}
\section{\sc Programming} Production-ready ML models (JAX, PyTorch). Expert scientific programmer (Python). Fluent in development tools (Git, Docker, etc.).

% ---------------------------------------------------------------------------
% \vspace{-.35cm}
% \section{\sc Press/Talks}
% Brain's `Background Noise' May Hold Clues to Persistent Mysteries, \emph{Quanta Magazine}, 2021. \\
% Build Your Own Brainwaves, \emph{Nerd Nite}, Los Angeles, Feb 2018. \\
% % Conflicted Data Science, \emph{Open San Diego}, San Diego, Feb, 2016. \\
% In Theory You're Paying Attention, \emph{Ignite}, San Diego, Nov 2016. \\
    
% ---------------------------------------------------------------------------
% \newpage
\vspace{-.4cm} 
\section{\sc Select Publications}
\textsc{Total citations}: $>$2,000. \textsc{h-index}: 14.
\\ 
\vspace{-.4cm} 
\\
\textbf{Peterson EJ} \& Lavin A, Physical Computing for Materials Acceleration Platforms, \textit{Matter} 5, 3586-3596 (2022).
% -- LONG FORMAT -- 
% \\ 
% \vspace{-.35cm} 
% \\
% Lavin A, ..., \textbf{Peterson EJ}.. et al, Simulation Intelligence: Towards a New Generation of Scientific Methods, \emph{arXiv} 2112.03235 (2021).
% \\ 
% \vspace{-.35cm} 
% \\
% \textbf{Peterson EJ}, What Can Astrocytes Compute?, \emph{bioRxiv} 465192 (2021).
\\ 
\vspace{-.4cm} 
\\
Donoghue T*, Haller M*, \textbf{Peterson EJ}*, et al, Parameterizing Neural Power Spectra into Periodic and Aperiodic Components, \emph{Nature Neuroscience} 23, 1655-1665 (2020). 
\\ 
\vspace{-.4cm} 
\\
\textbf{Peterson EJ} \& Verstynen T, Curiosity Eliminates the Exploration-Exploitation Dilemma, \emph{bioRxiv} 671362v8 (2020). 
% -- LONG FORMAT -- 
% \\ 
% \vspace{-.35cm} 
% \\
% \textbf{Peterson EJ} \& Voytek B, Homeostatic Mechanisms May Shape the Type and Duration of Oscillatory Modulation, \emph{J Neurophys} 124[1] (2020).
% \\ 
\\
\vspace{-.4cm} 
\\
Gao RD, \textbf{Peterson EJ}, Voytek B, Inferring Synaptic Excitation/Inhibition Balance from Field Potentials, \emph{Neuroimage} 158 (2017).

% ---------------------------------------------------------------------------
\end{resume}
\end{document}